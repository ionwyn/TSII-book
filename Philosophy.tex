\section{Absurdism}

Source: \href{https://plato.stanford.edu/entries/camus/#SuiAbsHapMytSis}{\underline{Stanford Encyclopedia of Philosophy}}\\\

Many people believe that the most fundamental philosophical problem is this: what is the meaning of existence? That’s a question that Albert Camus dug into in his novels, plays, and essays.

His answer was perhaps a little depressing. He thought that life had no meaning, that nothing exists that could ever be a source of meaning, and hence there is something deeply absurd about the human quest to find meaning. Appropriately, then, his philosophical view was called (existentialist) absurdism.

What would be the point of living if you thought that life was absurd, that it could never have meaning? This is precisely the question that Camus asks in his famous work, The Myth of Sisyphus. He says, “There is only one really serious philosophical problem, and that is suicide.” He was haunted by this question of whether suicide could be the only rational response to the absurdity of life.

But why did he think life was inherently without meaning? Don’t people find meaning in many different ways?

Take religion. It certainly seems to provide comfort to many people, but this could not amount to genuine meaning for Camus because it involves an illusion. Either God exists or he doesn’t. If he doesn’t, then it’s obvious why he could not be the source of life’s ultimate meaning. But what if God does exist? Given all the pain and suffering in the world, the only rational conclusion about God is that he’s either an imbecile or a psychopath. So, God’s existence could only make life more absurd, not less.

Of course, God is not the only possible source of meaning to consider. Think of our relations to other people—our family, our friends, our communities. We love and care for others in this cruel world, and perhaps that’s why we continue to live. That’s what gives existence meaning.

The problem here is that everyone we know and love will die some day, and some of them will suffer tremendously before that happens. How is that anything but absurd?

Before everyone gets too depressed, let’s think about some possible solutions to the problem. Let’s assume, with Camus, the absurdity of the quest for meaning. Let’s assume that any route we attempt to find meaning in the world will be for naught. They are all dead ends, so to speak. How do we avoid the conclusion that suicide is the answer?

Consider Nietzsche’s approach. Like Camus, he thought that life was devoid of intrinsic meaning. But he thought we could give it a kind of meaning by embracing illusion. That's what we have to learn from artists, according to Nietzsche. They are always devising new “inventions and artifices” that give things the appearance of being beautiful, when they’re not. By applying this to our own lives, we can become “the poets of our lives.” Could this be a possible solution?

The solution Camus arrives at is different from Nietzsche’s and is perhaps a more honest approach. The absurd hero takes no refuge in the illusions of art or religion. Yet neither does he despair in the face of absurdity—he doesn't just pack it all in. Instead, he openly embraces the absurdity of his condition. Sisyphus, condemned for all eternity to push a boulder up a mountain only to have it roll to the bottom again and again, fully recognizes the futility and pointlessness of his task. But he willingly pushes the boulder up the mountain every time it rolls down.

You might wonder how that counts as a solution. Here’s what I think Camus had in mind. We need to have an honest confrontation with the grim truth and, at the same time, be defiant in refusing to let that truth destroy life. At the end of Myth, Camus says that we have to “imagine Sisyphus happy.”

Perhaps my imagination is limited, but I’m not sure I find that thought comforting. Exactly how does confronting the absurdity of his situation give Sisyphus a reason to keep going? Maybe it’s not supposed to be comforting. But maybe it’s all that there is.

So, what do you think? Is life truly absurd? If so, can there be any point in living?

In the end, I guess my own approach to life’s absurdity is similar to Peggy Lee’s, who says that “if that’s all there is, then let’s keep dancing. Let’s break out the booze and have a ball, if that’s all there is…”