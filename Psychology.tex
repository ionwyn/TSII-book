\section{Cognitive Biases}

Source: \href{https://www.businessinsider.com/cognitive-biases-that-affect-decisions-2015-8}{\underline{Business Insider}}\\\

While there are many \href{https://en.wikipedia.org/wiki/List_of_cognitive_biases}{\underline{cognitive biases}}, here are 20 of them worth remembering, listed alphabetically, with example for each:

\begin{itemize}
    \item \textbf{Anchoring bias}
    
    We are over-reliant on the first piece of information we hear.  In a salary negotiation, whoever makes the first offer establishes a range of reasonable possibilities in each person's mind.
    
    \item \textbf{Availability heuristic}
    
    We overestimate the importance of information that is available to us.  A person might argue that smoking is not unhealthy because they know someone who lived to 100 and smoked three packs a day.
    
    \item \textbf{Bandwagon effect}
    
    The probability of one person adopting a belief increases based on the number of people who hold that belief.  This is a powerful form of groupthink and is reason why meetings are often unproductive.
    
    % To add into that, I think the weigh an expert has on an opinion is more likely to be adopted by the passive listener.
    
    \item \textbf{Blind-spot bias}
    
    Failing to recognize our own cognitive biases is a bias in itself.  People notice cognitive and motivational biases much more in others than in themselves.
    
    \item \textbf{Choice-supportive bias}
    
    When we choose something, we tend to feel positive about it, even if that choice has flaws.  Like how you think your new headphone is awesome, even though it's probably uncomfortable to wear.
    
    \item \textbf{Clustering illusion}
    
    This is the tendency to see patterns in random events.  This is key to various gambling fallacies, like the idea that red is more or less likely to turn up on a roulette table after a string of reds.
    
    \item \textbf{Confirmation bias}
    
    We tend to listen only to information that confirms your preconceptions - one of the many reasons it's so hard to have an intelligent conversation about religion.
    
    \item \textbf{Conservatism bias}
    
    We favor prior evidence over new evidence or information that has emerged.  People were slow to accept the fact that the Earth was round because they maintained their earlier understanding that the planet was flat.
    
    \item \textbf{Information bias}
    
    The tendency to seek information when it does not affect action.  More information is not always better.  With less information, people can often make more accurate predictions.
    
    \item \textbf{Ostrich effect}
    
    The decision to ignore dangerous or negative information by "burying" one's head in the sand, like an ostrich.  Investors check the value of their holdings significantly less often during bad markets.
    
    \item \textbf{Outcome bias}
    
    Judging a decision based on the outcome - rather than how exactly the decision was made in the moment.  Just because you won a lot in Las Vegas doesn't mean gambling your money was a smart decision.
    
    \item \textbf{Overconfidence}
    
    Some of us are too confident about our abilities, and this causes us to take greater risks in our daily lives.  Experts are more prone to this bias than laypeople, since they are more convinced that they are right.
    
    \item \textbf{Placebo effect}
    
    When simply believing that something will have a certain effect on us causes it to have that effect.  In medicine, people given fake pills often experience the same physiological effects as people given the real medicine.
    
    \item \textbf{Pro-innovation bias}
    
    When a proponent of an innovation tends to overvalue its usefulness and undervalue its limitations.  Silicon Valley 101.
    
    \item \textbf{Recency}
    
    The tendency to weigh the latest information more heavily than older data.  Investors more often think the market will always look the way it looks today and make unwise decisions.
    
    \item \textbf{Salience}
    
    Our tendency to focus on the most easily recognizable features of a person or concept.  When you think about dying, you might worry about being mauled by a lion, as opposed to what is statistically more likely, like dying in a car accident.
    
    \item \textbf{Selective perception}
    
    Allowing our expectations to influence how we perceive the world.  An experiment involving a football game between students from two universities showed that one team saw the opposing team commit more infractions.
    
    \item \textbf{Stereotyping}
    
    Expecting a group or person to have certain qualities without having real information about the person.  It allows us to quickly identify strangers as friends or enemies, but people tend to overuse and abuse it.
    
    \item \textbf{Survivorship bias}
    
    An error that comes from focusing only on surviving examples, causing us to misjudge a situation.  For instance, we might think that being an entrepreneur is easy because we haven't heard of all those who failed.
    
    \item \textbf{Zero-risk bias}
    
    We love certainty, even if it's counterproductive.  Eliminating risk entirely means there is no chance of harm being caused.
\end{itemize}

\section{7 Deadly Sins of Memory}

Source: \textit{The Seven Sins of Memory by Daniel L. Schacter}

\begin{enumerate}
    \item \textbf{Transience}: Information loses accessibility over time.  We often forget the plot of a book we read a while ago.
    
    \item \textbf{Blocking}: Inability to remember needed information.  Forgetting someone's name.
    
    \item \textbf{Absentmindedness} Inattention or shallow processing results in weak storage of information.  Not paying attention during lectures.
    
    \item \textbf{Misattribution}  When memory points to a wrong source.  Thinking that your friend was wearing a black shirt when she was actually wearing a yellow one.
    
    \item \textbf{Suggestibility}  Altering a memory because of misleading or distorting information.  Developing a false memory such as falsely remembering that your friend has a Nike shoes after watching a Nike commercial.
    
    \item \textbf{Bias} Influence of current knowledge on our memory for past events.  Thinking that you know the answer to a question after being told the answer.
    
    \item \textbf{Persistence}  The resurgence of unwanted or disturbing memories that we would like to forget.  Remembering a breakup that led you to depression.
\end{enumerate}